\chapter{Critical Evaluation}
\section{Comparison of SwarmAvg to Other Methods}
Although SwarmAvg has demonstrated promising results, it remains important to conduct a comparative analysis against established methods. This will enable individuals seeking to incorporate distributed machine learning into their future projects to make an informed decision whether SwarmAvg is the algorithm that they should use.

\subsection{Comparison of SwarmAvg to FedAvg}
SwarmAvg has the significant benefit of being more fault tolerant than FedAvg. This is because FedAvg has a central server, which means that the whole network would stop functioning if the central server is stopped. Not only this, but if a node drops its connection to the server, that node is effectively ignored from training, which has a negative effect on performance. \\

However, SwarmAvg has some significant drawbacks. For example, SwarmAvg has been shown to be slower to converge than FedAvg. Not only this, but SwarmAvg can create a significantly higher amount of network traffic than FedAvg or FL in general. This effect is maximised when every node is connected to every other node, meaning that the number of connections scale by $O(n^2)$, where $n$ is the number of nodes. FL will always both number of connections and network traffic scale with $O(n)$, as every node only needs a single connection to the server. However, it is worth noting that SwarmAvg will consume less network traffic in less dense networks of nodes. \\

An important advantage of SwarmAvg over FedAvg is that former can be used in sparsely connected networks without any modification. This means that it is possible for it to SwarmAvg to run even when most nodes in the network are not connected to most other nodes, which may be a more realistic scenario for certain low powered networks of devices such as IoT networks or robotic swarms. For this reason, SwarmAvg may be a better choice than FedAvg for these situations.

\subsection{Comparison of SwarmAvg to SwarmBC}
Despite the fact that this paper did not perform tests on SwarmBC, it is still possible to theoretically compare both SwarmAvg and SwarmBC, based on a comprehensive understanding of the inner workings of each. \\

One of the primary drawbacks of utilizing SwarmAvg as opposed to SwarmBC is the potential occurrence of divergent training during SwarmAvg training. The detrimental effect of divergent training may persist throughout the entire training process, leading to significantly worse performance than anticipated. Several factors can lead to divergent training, with low settings for $\alpha$ and $\gamma$, coupled with a sparsely-connected network, being highly correlated with its occurrence during testing. SwarmBC does not suffer from this issue, as the blockchain ensures that all nodes agree on the same model. However, during testing, divergent training was a rare phenomenon, and its remedy was a simple parameter tuning process upon detection. \\

One drawback of utilizing the SwarmAvg algorithm, as opposed to SwarmBC, is the absence of security features. The utilization of blockchain technology enables the straightforward authentication of nodes utilizing different algorithms that are frequently well-documented and integrated into the blockchain framework. On the other hand, SwarmAvg lacks any built-in authentication mechanism. As a result, to employ it securely, a separate authentication system would need to be implemented to prevent malicious parties from interfering with training. \\

SwarmAvg surpasses SwarmBC in terms of computational efficiency as nodes are not mandated to verify transactions with a blockchain, which involves a comparatively substantial overhead as opposed to simply averaging the models of a nodes neighbours. Consequently, SwarmBC necessitates higher processing power to operate for the same amount of training steps. This overhead could impede the speed of training on a resource-limited system, such as a swarm of drones, as a considerable amount of time is consumed in validating transactions, rather than performing training. Thus, SwarmAvg may be a more fitting choice than SwarmBC for swarms where processing power is a concern.

\section{Evaluation of this Study}
It must be acknowledged that this study presented certain limitations that necessitate careful consideration. The primary concerns are outlined as follows:

\begin{description}
	\item[Small Simulated Number of Nodes: ] A significant concern regarding this study is the limited number of nodes simulated during testing. Several drawbacks of FL in comparison to SL, such as server bottlenecking, are not apparent when testing on a limited number of nodes.
	
	\item[Lack of Testing Overheads: ] In the course of testing the SwarmAvg algorithm, it was determined that the assessment of its performance should be based on training steps rather than time. Although it would have been preferable to evaluate performance in relation to time, this was not feasible due to limitations in resources. The GPUs utilized in this study are primarily optimized for running a single GPU program effectively, whereas this research required multiple nodes to perform training simultaneously. As a result, it was noted during testing that training times varied significantly, rendering time-based measurement impractical due to high levels of noise.
	
	\item[Lack of Simulated Internet Lag: ] The algorithm proposed is designed for usage over the internet. However, the study solely evaluated its performance within a simulated environment, which lacked a time gap between the sending of information from one node to its reception by another. The deliberate omission of this time gap in the simulation aimed to increase the speed of simulations. However, this critical aspect of the algorithm's functionality remains untested as a result.
\end{description}