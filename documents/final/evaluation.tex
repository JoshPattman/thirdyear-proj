\section{Comparison of SwarmAvg to Other Methods}
\todo{Write something useful here}

\subsection{Comparison of SwarmAvg to FedAvg}
\todo{Write this out}

\begin{itemize}
	\item Can prevent a single slow node from bottlenecking the whole process
	\item No central server so more fault tolerant to node dropouts
	\item Can still function if all nodes do not have direct connections to all other nodes (in the case of FL every node must have access to the server)
\end{itemize}


\begin{itemize}
	\item Less data transmission ($O(n)$ rather than worst case $O(n^2)$). This effect is less prominent if SL is sparse
	\item FL is a Less complex algorithm -> fewer parameters to tune -> easier to fit to a problem
	\item Slower to converge
\end{itemize}

\subsection{Comparison of SwarmAvg to SwarmBC}
\todo{cant really compare results but can compare use of blockchain theoretically}

SwarmAvg can diverge, unlike using blockchain

SwarmAvg needs an extra layer of security, but this is built into blockchain

SwarmAvg might is faster and more lightweight due to the omission of a blockchain

SwarmAvg in theory may actually allow semi separated islands of nodes to learn slightly different models without fully diverging - this may be beneficial if the distribution of data changes. Cant do this with blockchain

\section{Pitfalls of this Study}

\subsection{The Small Simulated Number of Nodes}
A significant concern regarding this study is the limited number of nodes simulated during testing. Several drawbacks of FL in comparison to SL, such as server bottlenecking, are not apparent when testing on a limited number of nodes. Although SwarmAvg may have the potential to be utilized for training in large swarms, this particular scenario has not yet been evaluated.

\subsection{The Lack of Testing Overheads}
In the course of testing the SwarmAvg algorithm, it was determined that the assessment of its performance should be based on training steps rather than time. Although it would have been preferable to evaluate performance in relation to time, this was not feasible due to limitations in resources. The GPUs utilized in this study are primarily optimized for running a single GPU program effectively, whereas this research required multiple nodes to perform training simultaneously. As a result, it was noted during testing that training times varied significantly, rendering time-based measurement impractical due to high levels of noise.

\subsection{The Lack of Simulated Internet Lag}
The algorithm proposed is designed for usage over the internet. However, the study solely evaluated its performance within a simulated environment. It is worth noting that a fundamental distinction between the simulated environment and the real world is the absence of a time gap between the sending of a model update by one node and its reception by another. The deliberate omission of this time gap in the simulation aimed to increase the speed of simulations. However, this critical aspect of the algorithm's functionality remains untested as a result.