\section{Comparison of SwarmAvg to Other Methods}
Although SwarmAvg has demonstrated promising results, it remains important to conduct a comparative analysis against established methods. This will enable individuals seeking to incorporate distributed machine learning into their future projects to make and informed decisions whether SwarmAvg is the algorithm that they should use.

\subsection{Comparison of SwarmAvg to FedAvg}
\todo{Write this out}

\begin{itemize}
	\item Can prevent a single slow node from bottlenecking the whole process
	\item No central server so more fault tolerant to node dropouts
	\item Can still function if all nodes do not have direct connections to all other nodes (in the case of FL every node must have access to the server)
\end{itemize}


\begin{itemize}
	\item Less data transmission ($O(n)$ rather than worst case $O(n^2)$). This effect is less prominent if SL is sparse
	\item FL is a Less complex algorithm -> fewer parameters to tune -> easier to fit to a problem
	\item Slower to converge
\end{itemize}

\subsection{Comparison of SwarmAvg to SwarmBC}
Despite the fact that this paper did not perform tests on SwarmBC, it is still possible to compare the theoretical advantages of both methods over one another, based on a comprehensive understanding of the inner workings of each.

One of the primary drawbacks of utilizing SwarmAvg as opposed to SwarmBC is the potential occurrence of divergent training, which arises when individual nodes or groups of nodes begin to learn significantly different models from each other during SwarmAvg execution. Consequently, when merging the models from these groups, they prove incompatible and cause a decrease in overall performance. This detrimental effect may persist throughout the entire training process, leading to significantly worse outcomes than anticipated. Several factors can lead to divergent training, with low settings for $\alpha$ and $\gamma$, coupled with a sparsely-connected network, being highly correlated with its occurrence during testing. SwarmBC does not suffer from this issue, as the blockchain ensures that all nodes agree on the same model. However, during testing, divergent training was a rare phenomenon, and its remedy was a simple parameter tuning process upon detection.

One drawback of utilizing the SwarmAvg algorithm, as opposed to SwarmBC, is the absence of security features. The utilization of blockchain technology enables the straightforward authentication of nodes utilizing different algorithms that are frequently well-documented and integrated into the blockchain framework. On the other hand, SwarmAvg lacks any built-in authentication mechanism. As a result, to employ it securely, a separate stand-alone authentication system would need to be implemented. Failure to utilize authentication could result in a malicious node gaining entry and compromising the training process.

The rationale behind the development of SwarmAvg was to establish a more agile and efficient SL algorithm in comparison to the existing technologies. SwarmAvg surpasses SwarmBC in terms of computational efficiency, as nodes are not mandated to verify transactions. In contrast, Blockchain involves a comparatively substantial overhead as opposed to simply averaging the models of a nodes neighbours. Consequently, SwarmBC necessitates higher processing power to operate for the same amount of training. This overhead can impede the speed of training on a resource-limited system, such as a swarm of drones, as a considerable amount of time is consumed in validating transactions, rather than performing training. Thus, SwarmAvg may be a more fitting choice than SwarmBC for swarms where processing power is a concern.


\todo{SwarmAvg in theory may actually allow semi separated islands of nodes to learn slightly different models without fully diverging - this may be beneficial if the distribution of data changes. Cant do this with blockchain}

\section{Pitfalls of this Study}

\subsection{The Small Simulated Number of Nodes}
A significant concern regarding this study is the limited number of nodes simulated during testing. Several drawbacks of FL in comparison to SL, such as server bottlenecking, are not apparent when testing on a limited number of nodes. Although SwarmAvg may have the potential to be utilized for training in large swarms, this particular scenario has not yet been evaluated.

\subsection{The Lack of Testing Overheads}
In the course of testing the SwarmAvg algorithm, it was determined that the assessment of its performance should be based on training steps rather than time. Although it would have been preferable to evaluate performance in relation to time, this was not feasible due to limitations in resources. The GPUs utilized in this study are primarily optimized for running a single GPU program effectively, whereas this research required multiple nodes to perform training simultaneously. As a result, it was noted during testing that training times varied significantly, rendering time-based measurement impractical due to high levels of noise.

\subsection{The Lack of Simulated Internet Lag}
The algorithm proposed is designed for usage over the internet. However, the study solely evaluated its performance within a simulated environment. It is worth noting that a fundamental distinction between the simulated environment and the real world is the absence of a time gap between the sending of a model update by one node and its reception by another. The deliberate omission of this time gap in the simulation aimed to increase the speed of simulations. However, this critical aspect of the algorithm's functionality remains untested as a result.