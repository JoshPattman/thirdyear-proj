\chapter{Introduction, Problems and Goals}
Machine learning is becoming an exceedingly vital tool for our society to progress. However, many modern machine learning algorithms require large volumes of diverse data to achieve optimal performance. In the ideal world, this data would be stored in a single location close to a very powerful computer for training. Unfortunately, real-world data is often distributed among multiple nodes that are unable to share the data with each other or a central location, due to privacy regulations such as GDPR \cite{gdpr}. Accessing a super computer for training is also a luxury that many cannot afford. The reduction in data volume available can negatively impact the post training performance \cite{data_volume}, and the use of a slower computer means that training may not be able to be performed as fast as needed.

\section{Background}
It is evident that traditional machine learning techniques, which rely on a single machine, have certain limitations when utilized in practical applications. However, there are algorithms that aim to overcome these limitations by distributing the learning process. Moreover, a specific subset of these algorithms goes beyond distributed learning and achieves fully decentralized machine learning, which eliminates the need for a central server.

\section{Problems} \label{problems}
The below problems can arise from attempting to use single-machine learning techniques in the real world:
\begin{description}
	\item[Privacy:] It is common for data to be spread across multiple locations, referred to as data islands. Traditionally, all of this data would be consolidated into a single centralized server to facilitate the process of machine learning. However, it may not be possible to do so, given the potential conflict with privacy legislation. This leaves two options to the data scientist who is looking to train a model: either train a single model per data island, likely with inferior performance, or use an algorithm that would allow the different data islands to collaborate and collectively train a model.
	
	Consider the scenario where many different hospitals wish to train a model to detect an illness in a patient. Due to the obligation to maintain patient confidentiality, the medical data of patients cannot be shared with any of the other hospitals. This means that, despite likely having superior performance, a model trained on all data across all hospitals is not feasible, as a consequence patients would not have the highest quality medical care possible.
	
	\item[Performance:] In general, it has been observed that larger machine learning models coupled with more data typically lead to better performance. However, the use of conventional approaches for training these large models necessitates the requirement of a powerful computer. Most entities such as businesses or hobbyists, however, do not have access to such a computer. Nevertheless, they may have access to multiple, lower-power computers. For instance, during non-working hours, a company may have hundreds of computers in its offices that are not being used, thus providing a pool of unused processing power.
\end{description}

\section{Goals} \label{goals}
The primary objectives of the project are delineated below. The successful completion of these objectives will be indicative of the project's overall success.
\begin{description}
	\item[(A) Design an Novel Algorithm to Perform Machine Learning in a Swarm:] In this project, the primary aim is to design a novel algorithm to perform learning in a fully decentralised manner. This algorithm should be designed to be use in swarms of agents.
	\item[(B) Implement the Algorithm:] The algorithm should be implemented in an easy-to-understand programming language with a high focus on readability, so that future work could easily expand upon it.
	\item[(C) Test Performance in Situations Where Current Aproaches Perform Well:] The algorithm should be tested in situations where a current approach can be applied. The algorithm should be able to be at least comparable to the current approaches in performance.
	\item[(D) Test Performance in Situations Where Current Aproaches Perform Badly:] To show that the algorithm has a potential use case, it must be shown to perform in situations where current approaches have issues. It will also be useful to show the proposed algorithm can function when certain current approaches cannot.
\end{description}