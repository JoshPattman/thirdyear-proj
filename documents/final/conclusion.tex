\section{Conclusion}

\todo{Conclude what is SwarmAvg}

\todo{Conclude the findings + advantages}

\todo{Conclude the disadvantages}

\todo{Conclude why you would want to use SwarmAvg instead of another method}

\section{Future Work}
In the future, a vital step would be to test SwarmAvg on a significantly larger number of nodes, exceeding the current 10 node limit. The aforementioned constraint was established due to limitations on the machine used for training. However, if a data scientist were to design and implement an internet-enabled back-end to the given code, it would be feasible to distribute numerous nodes among multiple training machines, thus enabling the simulation of larger swarms. It would be particularly intriguing to observe the performance of SwarmAvg in scenarios that entail extremely large swarms, but only involve minute amounts of data and transmissions exclusively between a few neighbours for each node who dynamically change over time - a common occurrence in swarm robotics.

A second aspect of future research that could enhance the potential of SwarmAvg involves conducting an in-depth examination of optimal parameters for different situations. For the present study, parameters had to be chosen heuristically to meet testing purposes within the time constraints, potentially resulting in suboptimal outcomes as compared to what could have been achievable.

Finally, the scenarios in which SwarmAvg was tested where each node had access to only a subset of the classes left much to be desired. SwarmAvg performed far worse than FedAvg in these situations, however it remains plausible that with future research and modifications, that SwarmAvg may be able to handle these situations better.