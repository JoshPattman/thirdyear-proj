\chapter{Conclusion and Future Work}
\section{Conclusion}
This study has presented a novel SL algorithm algorithm, called SwarmAvg, which enables completely distributed and decentralised machine learning to take place. The algorithm is similar to FL in that it allows each node in the network to hold its own confidential dataset which never needs to be shared with any other nodes. \\

It has been found that SwarmAvg is less performant than FedAvg in all situations where FedAvg can be applied. However, in many of these scenarios, SwarmAvg reached a similar final accuracy to FedAvg, but training took more iterations to reach this. The area where SwarmAvg showed most potential was when performing training in a sparsely connected network of nodes. This is a situation where FedAvg cannot function, but SwarmAvg saw only minor reductions in performance when transitioning to this scenario. Unfortunately, it was shown that SwarmAvg performs significantly worse than FedAvg in situations where each node only has access to a small subset of the total classes. Despite the fact that this is an extreme case, unevenly distributed data is an issue for many algorithms in the real world, possibly indicating that SwarmAvg may need further improvements before deployment. \\

Notwithstanding these disadvantages, SwarmAvg could potentially be a compelling algorithm to employ in particular circumstances. For instance, in a very large swarm, FL may become impossible to use due to the bottleneck of sending all data to a single node, a problem that SwarmAvg does not face. SwarmAvg also may be a better choice if it is known that the network in which the algorithm will be deployed to is very unreliable, with nodes and connections dropping out. \\

Overall, the goals specified in Section \ref{goals} have been met, so the project can ba considered a success:
\begin{description}
	\item[(A)] A novel algorithm for use with machine learning was successfully designed. It functioned in a completely decentralised manner.
	\item[(B)] The algorithm was successfully implemented.
	\item[(C)] The performance was tested in densely connected networks, with both access to the full dataset and access to a subset of the dataset's classes. It was compared to FedAvg.
	\item[(D)] The algorithm was shown to perform in situations where FedAvg could not be used.
\end{description}

\section{Future Work}
\begin{description}
	\item[Simulate More Nodes:] In this paper, only 10 concurrent nodes were tested, a constraint which was established due to limitations on the machine used for training. However, if a data scientist were to design and implement an internet-enabled back-end to the given code, it would be feasible to distribute numerous nodes among multiple training machines, thus enabling the simulation of larger swarms.
	\item[Study of Effects of SwarmAvg Parameters: ] For the present study, parameters had to be chosen heuristically to meet testing purposes within the time constraints, potentially resulting in suboptimal outcomes as compared to what could have been achievable. An in depth examinations into the effects of different parameters could be beneficial to future work.
	\item[Experimenting with a Hybrid FL-SL approach: ] In this study, FedAvg has been shown to be more effective in scenarios where the environment is tightly controlled when compared to SwarmAvg. However, a potential approach would multiple utilise small clusters of FedAvg, where the FedAvg servers are connected with SwarmAvg. This may inherit both the high performance of FedAvg, but with some of the fault tolerance of SwarmAvg.
	\item[Dynamic Environment: ] In all of the presented experiments, the simulated environments were static, meaning that the connections between nodes did not change throughout training. A dynamic environment may have significant advantages, as each node is exposed to more varied neighbours, which could lead to higher performance.
	\item[Improve Unevenly Distributed Class Performance: ] The scenarios in which SwarmAvg was tested where each node had access to only a subset of the classes left much to be desired. SwarmAvg performed far worse than FedAvg in these situations, however it remains plausible that with future research and modifications, that SwarmAvg may be able to handle these situations better.
\end{description}