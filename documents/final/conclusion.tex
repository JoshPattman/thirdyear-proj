\section{Conclusion}

\todo{Conclude what is SL}

\todo{Conclude the findings + advantages}

\todo{Conclude the disadvantages}

\section{Future Work}
In the future, a vital step would be to test \SL on a significantly larger number of nodes, exceeding the current 10 node limit. The aforementioned constraint was established due to limitations on the machine used for training. However, if a data scientist were to design and implement an internet-enabled backend to the given code, it would be feasible to distribute numerous nodes among multiple training machines, thus enabling the simulation of larger swarms. It would be particularly intriguing to observe the performance of \SL in scenarios that entail extremely large swarms, but only involve minute amounts of data and transmissions exclusively between a few neighbours for each node who dynamically change over time - a common occurrence in swarm robotics.

A second aspect of future research that could enhance the potential of \SL involves conducting an in-depth examination of optimal parameters for different situations. For the present study, parameters had to be chosen heuristically to meet testing purposes within the time constraints, potentially resulting in suboptimal outcomes as compared to what could have been achievable.

Finally, although the \SL algorithm is primarily designed for machine learning, its usability is not exclusive to that domain. The algorithm can be employed in any scenario where a group of nodes necessitates collaborative action to modify an array of parameters, without sharing anything bu the parameters between them. The author is certain that there are other uses of \SL not yet discovered.