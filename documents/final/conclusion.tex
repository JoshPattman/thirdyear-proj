\chapter{Conclusion and Future Work}
\section{Conclusion}

\todo{Conclude why you would want to use SwarmAvg instead of another method}

This study has presented a novel SL algorithm algorithm, called SwarmAvg, which enables completely distributed and decentralised machine learning to take place. The algorithm is similar to FL in that it allows each node in the network to hold its own confidential dataset which never needs to be shared with any other nodes.

It has been found that SwarmAvg is less performant than FedAvg in all situations where FedAvg can be applied. However, in many of these scenarios, SwarmAvg reached a similar final accuracy to FedAvg, but training took more iterations to reach this. The area where SwarmAvg showed most potential was when performing training in a sparsely connected network of nodes. This is a situation where FedAvg cannot be used, however SwarmAvg saw only minor reductions in performance when transitioning to this scenario. Unfortunately, it was shown that SwarmAvg performs significantly worse than FedAvg in situations where each node only has access to a small subset of the total classes. Despite the fact that this is an extreme case, unevenly distributed data is an issue for many algorithms in the real world, possibly indicating that SwarmAvg may need further improvements before deployment.

Notwithstanding these disadvantages, SwarmAvg could potentially be a compelling algorithm to employ in particular circumstances. For instance, in a very large swarm, FL may become impossible to use due to the bottleneck of sending all data to a single node, a problem that SwarmAvg does not face. SwarmAvg also may be a better choice if it is known that the network in which the algorithm will be deployed to is very unreliable, with nodes and connections dropping out.

Overall, the goals specified in Section \ref{goals} have been met. Most of the time, SwarmAvg also performed comparably to FedAvg. For these reasons, the author believes this project to be a success.

\section{Future Work}
In the future, a vital step would be to test SwarmAvg on a significantly larger number of nodes, exceeding the current 10 node limit. The aforementioned constraint was established due to limitations on the machine used for training. However, if a data scientist were to design and implement an internet-enabled back-end to the given code, it would be feasible to distribute numerous nodes among multiple training machines, thus enabling the simulation of larger swarms. It would be particularly intriguing to observe the performance of SwarmAvg in scenarios that entail extremely large swarms, but only involve minute amounts of data and transmissions exclusively between a few neighbours for each node who dynamically change over time - a common occurrence in swarm robotics.

A second aspect of future research that could enhance the potential of SwarmAvg involves conducting an in-depth examination of optimal parameters for different situations. For the present study, parameters had to be chosen heuristically to meet testing purposes within the time constraints, potentially resulting in suboptimal outcomes as compared to what could have been achievable.

Merge SL with FL

Dynamic agents

Finally, the scenarios in which SwarmAvg was tested where each node had access to only a subset of the classes left much to be desired. SwarmAvg performed far worse than FedAvg in these situations, however it remains plausible that with future research and modifications, that SwarmAvg may be able to handle these situations better.