\chapter{Project Management}
\section{Time Management}
To assist with planning and organisation of this project, Gantt charts were used. These helped to visualise which tasks needed doing, and also if the project progress was ahead or behind schedule. The Gantt for the interim report can be found in appendix \ref{gc-interim} and the final report Gantt can be found at \ref{gc-final}.

On the Gantt charts, it was decided to not count weekends as working days. This was chosen as it represented the fact that the author had other work to do during the course of the project. 

The presented charts represent the final iteration of planning. As described in the following section, changes occurred throughout the project and the Gantt charts were adapted accordingly.

\section{Changing Plans}
Initially, the project proposal included the implementation of both SL and a simulator. The intended purpose of the simulator was to generate camera feeds that simulated those of security and doorbell cameras from various locations in a neighbourhood. Subsequently, the SL model would be employed to learn the task of identifying car crashes. However, this project was deemed excessively ambitious, and as a result, a decision was taken to significantly streamline the project by discarding many of its components. \\

Plans were also changed on a smaller scale somewhat regularly, due to the experimental nature of the project. However, this was not an issue due to the inclusion of many buffer zones in the planned time allocation, such as \emph{Bugfixing Time}. This meant that an overrun task could be finished during this time, so it would not effect the project moving forward.

\section{Risk Assessment}
Below is the risk assessment table created at the beginning of the project:
\begin{table}[H]
	\begin{tabular}{p{4cm}|l|l|l|p{10cm}}
		Risk    & \rotatebox{90}{Severity}  & \rotatebox{90}{Likelihood} & \rotatebox{90}{Score} & Mitigation \\
		\hline \hline
		Personal issues such as illness prevent work from being done & 3        & 3          & 9     & The code will be designed such that there are many checkpoints with acceptable results. This means that even if the project is not finished, it will still have an satisfactory outcome. \\
		\hline
		Failure of authors computer & 4        & 2          & 8     & Project is backed up on Github. Development environment is backed up on google drive using docker. This means the author has the possibility to run the code on the Zepler labs computers using an identical environment. \\
		\hline
		Algorithm does not work as expected & 5        & 1          & 5     & During research, the author read a large amount of literature on Distributed Federated Learning (DFL). It would be possible to design a DFL algorithm and perform tests on that instead, which is more likely to work.   
	\end{tabular}
\end{table}