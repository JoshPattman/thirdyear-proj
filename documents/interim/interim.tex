\documentclass[12pt,a4paper,titlepage]{report}
\usepackage[utf8]{inputenc}
\usepackage[T1]{fontenc}
\usepackage{amsmath}
\usepackage{amssymb}
\usepackage{graphicx}
\usepackage{cite}
\usepackage{subcaption}
\graphicspath{ {./figures/} }

\newcommand{\citeme}[0]{\textbf{\emph{CITEME}}}
\newcommand{\todo}[1]{\textbf{\emph{TODO: } #1}}

\usepackage{titlesec}

\titleformat{\chapter}[display]{\normalfont\bfseries}{}{0pt}{\Huge}


\begin{document}
	\begin{titlepage}
		\centering\Large\emph{Electronics and Computer Science Faculty of Engineering and Physical Sciences University of Southampton}
		\\[2cm]
		\centering\Large{Josh Pattman} \\
		\centering\Large{12 December 2022} \\
		\centering\huge\textbf{Investigating Optimisations Of Swarm Learning With Respect To Real World Challenges}
		\\[2cm]
		\centering{\large{Project Supervisor:} \linebreak \Large{Dr Mohammad Soorati - \emph{m.soorati@soton.ac.uk}}} \\[1cm]
		\centering{\large{Second Examiner:} \linebreak \Large{Dr Jo Grundy - \emph{j.grundy@soton.ac.uk}}}
		\\[2cm]
		\centering\Large{A project progress report submitted for the award of \textbf{Computer Science with Artificial Intelligence}}
	\end{titlepage}

	\chapter*{Statement of Originality}
\begin{itemize}
	\item I have read and understood the ECS Academic Integrity information and the University’s
	Academic Integrity Guidance for Students.
	\item I am aware that failure to act in accordance with the Regulations Governing Academic Integrity
	may lead to the imposition of penalties which, for the most serious cases, may include
	termination of programme.
	\item I consent to the University copying and distributing any or all of my work in any form and
	using third parties (who may be based outside the EU/EEA) to verify whether my work
	contains plagiarised material, and for quality assurance purposes.

	\item I have acknowledged all sources, and identified any content taken from elsewhere.
	\item I have not used any resources produced by anyone else.
	\item I did all the work myself, or with my allocated group, and have not helped anyone else.
	\item The material in the report is genuine, and I have included all my data/code/designs.
	\item I have not submitted any part of this work for another assessment.
	\item My work did not involve human participants, their cells or data, or animals.

\end{itemize}


	\chapter*{Abstract}	
	Federated learning is a technique that allows a machine learning model to be trained on data distributed across multiple data islands without ever sharing the data. This approach protects privacy by keeping the data decentralized. Swarm learning is a similar technique that eliminates the need for a central server, ensuring that not only the data, but also the communication, is completely decentralized. The goal of this project is to design and implement a swarm learning algorithm, then test the algorithm in multiple scenarios inspired by the real world. The algorithm will then be evaluated against other machine learning techniques to investigate if swarm learning is a viable technology for the modern world.

	\tableofcontents
	\chapter{Background}
\section{Background Research}
\subsection{Federated Learning}
% Introduce federated learning
Many modern machine learning algorithms require vast volumes of diverse data to perform optimally. Real world data is often distributed around multiple organisations who are not able to share the data between them due to privacy regulations, which can limit machine learning models from reaching their full potential. Federated Learning (FL) \cite{survey_on_fed_learning} is a technique in machine learning that aims to train a single model using all of the available data across organisations, without ever requiring data to be shared between parties. \\

% Intro on how fed learning works and different types
There are a multitude of published FL frameworks with different merits and drawbacks \cite{survey_on_fed_learning}[p.~12]. Federated Averaging (FedAvg) \cite{fed_learning} is a commonly used yet simple framework, which breaks training into iterations where three steps take place:
\begin{enumerate}
	\item A copy of the current model is sent to each agent
	\item Each agent performs some training with their copy of the model and their own private data
	\item The trained models from each agent are sent back to the server to aggregate into a global model
\end{enumerate}
The global model improves over time, and after a certain number of steps, the training is complete and the global model is the final output. \\

TODO: \cite{simfl}
% Specify why federated learning is better than conventional learning
As it does not require data to be shared between agents, FL is naturally beneficially for privacy sensitive tasks compared to conventional machine learning where the data is aggregated in a central location. However, as FL performs training on multiple agents in parallel, TODO FIX ME: it can be cheaper to implemnt/more efficient with available resources.

\subsection{Swarm Learning}
% Introduce swarm learning
Swarm learning (SL) \cite{swarm_learning} is a subcategory of FL which operates in a completely distributed and decentralised manner. It differs from decentralised FL techniques such as SimFL \cite{simfl} by not using a leader election technique. In contrast, SL methods usually use a blockchain based system to coordinate an agreed upon global model. SL does however still work in training iterations, but each agent does not have to perform iterations at the same time:
\begin{enumerate}
	\item An agent obtains a copy of the current model from the blockchain
	\item The agent performs some training with their copy of the model and their own private data
	\item The updated model is merged with the current model on the blockchain and sent back to the blockchain for other agents to use
\end{enumerate}
The model on the blockchain will improve over time, and after a certain number of steps the blockchain model is the final output. \\

% Specify why swarm learning is better than conventional learning or federated learning
SL shares many of the same benefits of FL over conventional learning, however it also improves upon FL in some respects. The dispensing of a central server means that SL is more robust to failures than centralised FL such as FedAvg \cite{fed_learning}. The lack of a leader election protocol also means that SL may be better suited to tasks where networks of agents are sparsely connected, as data would not be needed to be forwarded around the network as frequently, as opposed to distributed FL techniques such as SimFL \cite{simfl} CITE ME.

\subsection{Challenges for Federated and Swarm Learning}
There are many challenges that one may face when trying to implement FL or SL as opposed to a conventional learning system. However, many of the proposed algorithms are designed for FL, and have not been adapted to work or tested with SL yet.

\subsubsection{Unevenly distributed features/local bias}
\cite{multi_center_fed_learning}
\cite{fedbn}
\cite{model_agnostic_meta_learning}

\subsubsection{Training speed}

\subsubsection{Sparsely connected networks}

\subsubsection{Data transfer limits}
In the real world, it is sometimes not possible to transfer entire models between agents on a regular basis, due to data transfer limits. It has been proposed that a possible solution to use FedAvg \cite{fed_learning}, but only transfer segments of the network at each training step \cite{gossip_learning}. This approach outperformed FedAvg by a factor of between 2.25 and 3.01 with respect to training speed, caused by the removal of the central server bottleneck and reduced data transferred \cite{gossip_learning}[section.~5.2].

\subsubsection{Low processing power agents}
\cite{fed_in_robotics}

	\chapter{Project Goals}
\section{Problem}
Currently, there is more data stored around the world than ever before. However, there are also many recent regulations that prevent the data from being used by machine learning algorithms. In many cases, this is because the data owners are not comfortable or able to send the data to a central server for processing and training. The result of this is that either each data owner has to train their own inferior models, or no models are trained at all due to the lack of data.

In addition to this, the world also has more electronic devices than ever before. As these devices become more and more powerful, using each device to train a machine learning model on a small amount of data becomes a possibility. However, these devices will still not outperform a fast computer alone.

\section{Proposed Solution}
Swarm learning is a type of machine learning which uses multiple devices or agents to train a machine learning model. It is similar to federated learning in this repect. However, swarm learning is a completely decentralised algorithm, where the agents decide how to train. This contrasts federated learning which requires a central server to work. One of the main features of swarm learning is that data never has to be shared amongst the agents in the learning network, which is excellent for the protection of private data. For this reason, swarm learning seems to fit the problem well, but there is another issue with swarm learning: it is still a relatively new algorithm and it faces many real world challenges that have only been adressed in isolation by existing research.

\section{Goals}
\textbf{To investigate possible optimisations of the swarm learning algorithm on simple problems, whilst adding real world constraints incrementally, and eventaully arriving at a robust swarm learning algorithm that adresses many real world issues in one.}

\begin{enumerate}
	\item Create a working implementation on a simple dataset of swarm learning
	\item Incrementally add real world problems to the swarm learning algorithm
	\item Mitigate the effects of the problem on the accuracy of the algorithm by implementing either novel ideas or methods from existing literature.
	\item Ensure that the end product is as reproducable as possible such that it can be re-implemented for specific use cases.
\end{enumerate}
\subsubsection{Problems}
Below are some real world problems with swarm learning. This is not a comprehensive list and more problems could be researched if nescesary.
\begin{itemize}
	\item \textbf{Unevenly distributed features/local bias} - This is where feature in the dataset are not distributed evently between agents. For instance, when classifying mnist, one agent may see more 7s and one may see more 2s.
	\item \textbf{Sparsely connected networks} - This refers to a situation where each agent does not have direct communication with every other agent, but instead can only communicate with a few neighbors.
	\item \textbf{Data transfer limits} - This is a situation where an agent may not be able to send an entrire neural network over the internet, due to connection speed.
	\item \textbf{Low processing power agents} - This situation is where agents have much lower processing power than a standard machine that one may train a model on. For example, agents may not have accsess to a GPU.
\end{itemize}	

\section{Focussing On Project Goals}
The plan for this project went through multiple iterations before the final plan was formulated:
\begin{enumerate}
	\item The initial plan was to \emph{control a swarm of drones to detect objects such as natural disasters or people needing help, whilst also improving accuracy of the model over time}
	\item After discussion, the plan was changed to be \emph{perform edge processing and distributed object detection on many camera perspectives of an environment to decide where disasters are happening, whilst learning to improve the model over time}
	\item Following the initial research phase, the plan was narrowed to \emph{explore ways to optimise swarm learning for distributed detection of a simple abstract object}
	\item Finally, after the second phase of research, the settled upon plan became \emph{Investigate possible optimisations of the swarm learning algorithm}. This is the current plan.
\end{enumerate}
The shift of focus away from object detection and towards swarm learning is mainly for two reasons:
\begin{enumerate}
	\item The author finds the swarm learning aspect of the project to be the most interesting part, especially after researching the subject
	\item A general framework of improvements and algorithms on swarm learning is much more useful to the real world than an implementation on a simple simulation.
\end{enumerate}

\section{Risk Assessment}
\subsection{Personal Issues}
\subsubsection{Description}
This risk entails all personal issues which cause the author to be unable to do work, such as illness.
\subsubsection{Risk Calculations}
\emph{Severity (1-5):} 3 \\
\emph{Likelihood (1-5):} 3 \\
\emph{Overall Risk (1-25):} \textbf{9}
\subsubsection{Mitigation}
As many sections and modules as possible from the codebase will be designed to have minimal requirements from other sections. This means that, even if the author is unable to work for a period of time, some less important sections can be skipped with minimal effect on the reset of the project.
\subsection{Hardware Failure - Local Computer}
\subsubsection{Description}
This risk entails a failure on the authors local computer of any kind, such as a graphics card or storage breakage.
\subsubsection{Risk Calculations}
\emph{Severity (1-5):} 4 \\
\emph{Likelihood (1-5):} 2 \\
\emph{Overall Risk (1-25):} \textbf{8}
\subsubsection{Mitigation}
To mitigate storage based failures, the project will be regularly backed up to \emph{GitHub}. If a core component of the work computer breaks, the author has access to a personal laptop and the \emph{Zepler Labs}. The deep learning environment along with dependencies is backed up to the authors \emph{Google Drive} in the form of a docker image, so that switching to a new computer would be a smooth process.
\subsection{Hardware Failure - Iridis 5}
\subsubsection{Description}
This risk entails a failure on the \emph{Iridis 5 Compute Cluster} which prevents it from being accessed by the author.
\subsubsection{Risk Calculations}
\emph{Severity (1-5):} 5 \\
\emph{Likelihood (1-5):} 1 \\
\emph{Overall Risk (1-25):} \textbf{5}
\subsubsection{Mitigation}
\emph{Iridis 5} will play a key role in this project when simulating large numbers of agents at once. However, it is possible to simulate lower numbers (around 10) agents at the same time on the authors local machine with a basic dataset. This could be a temporary solution if \emph{Iridis 5} went down for a short time. However, for a more permanent solution, funding may be acquired from the university to run the project on a cluster of \emph{AWS} servers, as the author has some experience in that field. 
	\chapter{Progress of Implementations}

\section{Basic MNIST classifier}
A simple \emph{mnist} classifier was built using \emph{Keras} in \emph{Python}. There was only one agent which trained on the dataset, and the main reason for this was to get a baseline for the swarm learning to compete against. This solution was trained on a single \emph{GTX 1660}.

\section{Simple swarm learning}
Using the same model for each agent as used in the \emph{Basic MNIST classifier}, a swarm of agents was created. Each agent had access to the whole dataset for training. Every agent also could communicate with every other agent. The agents acted in a loop, where they would each first train for one epoch on their copy of the training set, and then would average their models weights with all of their neighbours weights. This implementation consisted of 5 agents, all of whom had to share a single \emph{GTX 1660} between them. The following graphs showing the multiple agent accuracy show only the accuracy of one agent in the swarm, but all of the agents had similar results.

\includegraphics[width=\textwidth]{accuracy_60k}

To start off with, the swarm was a lot slower to increase its accuracy than the single agent algorithm, which is likely due to the higher overhead of running swarm learning. After some time, the swarm increased in accuracy to the same point as the single agent. However, after reaching this point, the single agent started to overfit but the swarm continued to increase in accuracy. The swam eventually settled upon a higher accuracy than the single agent ever achieved.

The agents seemed to reach the final accuracy in fewer training steps if the agents training loops were offset by even intervals of time. The author hypothesizes that this may be because when the agents training is synced, some of the agents may skip the just completed training when requesting updates, which can cause training epochs to effectively be lost. This effect needs further investigation.

\section{Reduced dataset swarm learning}
This experiment was focussed around reducing the amount of data each agent gets to train on. A number of samples were selected randomly for each agent from the training set, and these never changed throughout training. This test was run for 2000 and 500 samples per agent.

\includegraphics[width=\textwidth]{accuracy_2k}

\includegraphics[width=\textwidth]{accuracy_500}

In both cases, the swarm took a lot longer to reach its peak accuracy, nevertheless, the swarm exceeded the peak accuracy of the lone agent by about 5\%-10\%.

On both graphs, there is a spike in accuracy near the start of swarm training, then a reduction in accuracy, then a stable increase. This is caused by new agents with untrained networks joining the swarm and adding effectively a random model to the pool of weights. An easy fix for this would to make an agent that has just joined the network request a model from its neighbours immediately, and use that as a base network. This has not been implemented due to time constraints. 
	\chapter{Project Planning}

\section{Planned Phases of the Project}
The project is split up into a number of phases, each of which must be completed sequentially. During these phases the interim report and final report will be developed as a skeleton in parralell.
\begin{enumerate}
	\item Research - This phase consists of finding and reading existing literature. It is at this time that the project idea is developed fully.
	\item Simple Implementation - In this phase, a proof of concept implementation will be deveolped. It will entail a simple swarm learning algorithm learning on a basic dataset, without any complex problems. This implementation will be build upon in further phases.
	\item Interim Report - This phase is focussed on filling out and finalising the interim report.
	\item Main development - In this phase, the iterative development of the swarm learning algorithm will occur. This is actually a repetition of four sub-phases:
	\begin{enumerate}
		\item Further research on real-world problem
		\item Implement real-world problem and test current solution
		\item Implement mitigations and test / evaluate
		\item Document any discoveries and evaluations of mitigations
	\end{enumerate}
	\item Final report - In this phase, the final report will be written up.
	\item Housekeeping - This is the final phase which is used for any extra jobs that were not able to be completed in other phases.
\end{enumerate}



\section{Completed Work}
\includegraphics[width=\textwidth]{gantt_pre}



\section{Remaining Work}
INSERT GANT CHART HERE



\section{Risk Assessment}
\subsection{Personal Issues}

\subsubsection{Description}
This risk entails all personal issues which cause the author to be unable to do work, such as illness.

\subsubsection{Risk Calculations}
\emph{Severity (1-5):} 3 \\
\emph{Likelihood (1-5):} 3 \\
\emph{Overall Risk (1-25):} \textbf{9}

\subsubsection{Mitigation}
As many sections and modules as possible from the codebase will be designed to have minimal requirements from other sections. This means that, even if the author is unable to work for a period of time, some less important sections can be skipped with minimal effect on the reset of the project.

\subsection{Hardware Failure - Local Computer}
\subsubsection{Description}
This risk entails a failure on the authors local computer of any kind, such as a graphics card or storage breakage.

\subsubsection{Risk Calculations}
\emph{Severity (1-5):} 4 \\
\emph{Likelihood (1-5):} 2 \\
\emph{Overall Risk (1-25):} \textbf{8}

\subsubsection{Mitigation}
To mitigate storage based failures, the project will be regularly backed up to \emph{GitHub}. If a core component of the work computer breaks, the author has access to a personal laptop and the \emph{Zepler Labs}. The deep learning environment along with dependencies is backed up to the authors \emph{Google Drive} in the form of a docker image, so that switching to a new computer would be a smooth process.

\subsection{Hardware Failure - Iridis 5}
\subsubsection{Description}
This risk entails a failure on the \emph{Iridis 5 Compute Cluster} which prevents it from being accessed by the author.

\subsubsection{Risk Calculations}
\emph{Severity (1-5):} 4 \\
\emph{Likelihood (1-5):} 1 \\
\emph{Overall Risk (1-25):} \textbf{4}

\subsubsection{Mitigation}
\emph{Iridis 5} will play a key role in this project when simulating large numbers of agents at once. However, it is possible to simulate lower numbers (around 10) agents at the same time on the authors local machine with a basic dataset. This could be a temporary solution if \emph{Iridis 5} went down for a short time. However, for a more permanent solution, funding may be acquired from the university to run the project on a cluster of \emph{AWS} servers, as the author has some experience in that field. 

\subsection{Algorithm Fails to Work}
\subsubsection{Description}
This risk describes a situation where the algorithm of swarm learning does not function as well as expected. However, this is very unlikely as swarm learning has been proven to work in numerous papers CITE ME.

\subsubsection{Risk Calculations}
\emph{Severity (1-5):} 5 \\
\emph{Likelihood (1-5):} 1 \\
\emph{Overall Risk (1-25):} \textbf{5}

\subsubsection{Mitigation}
This is not an ideal situation, however it may be possible to shift the project away from swarm learning and onto federated learning. Federated learning is more commonly used and therefore has more literature, meaning that it is more likely to be an achievable goal to implement it. This would be a last resort though.
	\chapter{Conclusion}
	In conclusion, this paper has shown that SL has the potential to outperform conventional machine learning methods in some cases. While SL presents its own challenges, a plan has been proposed to address these issues and prepare SL for real-world applications. Preliminary testing has been encouraging, and future experiments may provide further evidence of the superiority of SL over conventional machine learning.
	
	\bibliographystyle{ieeetr}
	\bibliography{interim}{}
\end{document}