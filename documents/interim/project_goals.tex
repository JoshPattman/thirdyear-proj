\chapter{Project Goals}
\section{Problem}
Currently, there is more data stored around the world than ever before. However, there are also many recent regulations that prevent the data from being used by machine learning algorithms. In many cases, this is because the data owners are not legally able to send the data to a central server for processing and training. The result of this is that either each data owner has to train their own inferior models, or no models are trained at all due to the lack of data.

In short, the world contains vast amounts of data divided into data islands, with few efficient methods to processing the data. If used correctly, this data could improve lives.

\section{Proposed Solution}
One of the key features of SL is that data never needs to be shared among the agents in the learning network, which is beneficial for the protection of private data. Given this, SL seems well-suited to the problem at hand. In addition, SL may be able to more effectively utilize the processing power of a network of connected devices. However, SL is still a relatively new algorithm and faces many real-world challenges that have only been addressed individually in existing research.

\section{Goals}
To investigate possible optimisations of the SL algorithm on simple problems, whilst adding real world constraints incrementally, and eventually arriving at a robust SL algorithm that addresses many real world issues in one.

\begin{enumerate}
	\item Create an SL model on a simple dataset such as MNIST without any real world constraints
	\item Incrementally add real world problems (listed below) to the SL algorithm
	\item Mitigate the effects of each problem by implementing either novel ideas or methods from existing literature
	\item Ensure that the end product is as reproducible as possible such that it can be re-implemented for specific use cases
\end{enumerate}
\subsubsection{Real-Word Problems}
Below are some common real-world problems that effect SL:
\begin{itemize}
	\item \textbf{Unevenly distributed features/local bias} - Features in the dataset are not distributed evenly between agents. For instance, when classifying MNIST (a dataset that contains images of digits 0-9), one agent may see more 7s and one may see more 2s
	\item \textbf{Sparsely connected networks} - Each agent does not have direct communication with every other agent, but instead can only communicate with a few neighbours
	\item \textbf{Data transfer limits} - An agent may not be able to send an entire model over the internet, due to connection speed
	\item \textbf{Low processing power agents} - Agents have much lower processing power than a standard machine that one may train a model on. For example, agents may not have access to a GPU
\end{itemize}	

\section{Focussing On Project Goals}
At the outset, the project goal covered a very wide topic and was unachievable: It was to \emph{control a swarm of drones to detect objects such as natural disasters, whilst also improving accuracy of the model over time}. However, following multiple stages of research and meetings, the idea was focussed into a smaller, achievable goal. The shift in focus from object detection to SL was primarily motivated by the author's interest in the potential usefulness of a general framework for SL algorithms.