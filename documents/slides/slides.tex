\documentclass{beamer}
\usepackage[normalem]{ulem}
\usepackage{graphicx}

\usetheme{metropolis}

%Information to be included in the title page:
\title{Swarm Learning - A Fully Decentralised Approach To Machine Learning}
\author{Josh Pattman}
\institute{University Of Southampton}
\date{March 2023}
\graphicspath{{plots}}

\begin{document}
	\frame{\titlepage}
	
	\begin{frame}
		\frametitle{The Problems}
		\textbf{Privacy}
		\begin{itemize}
			\item Data stored in multiple locations
			\item Cannot share the data between locations for privacy reasons
			\item \emph{Medical records}
		\end{itemize}
		
		\textbf{Performance}
		\begin{itemize}
			\item Machine learning needs lots of processing power
			\item A supercomputer is not available to many
			\item However they may have access to many lower power devices (nodes)
			\item \emph{Company with many unused computers during the night}
		\end{itemize}
	\end{frame}

	\begin{frame}
		\frametitle{Federated Learning - The Current Solution}
		\begin{itemize}
			\item A single model is stored on the server
			\item Each node has its own dataset
			\begin{itemize}
				\item This is not shared with other nodes or the server
			\end{itemize}
			\item The model can be shared between the server and clients
			\item \textbf{Goal:} Perform machine learning without sharing the data
		\end{itemize}
	\end{frame}

	\begin{frame}
		\frametitle{Federated Learning - How Does It Work?}
		\begin{itemize}
			\item Many variations of federated learning
			\begin{itemize}
				\item One of the originals is \emph{Federated Averaging (FedAvg)}
				\item Many other algorithms are based off this
			\end{itemize}
			\item FedAvg has repeated training steps. Each step:
			\begin{enumerate}
				\item Server sends model to a set of nodes
				\item Nodes perform training on the model
				\item Nodes send their models back to server
				\item New model is the average (mean) of all nodes models
			\end{enumerate}
		\end{itemize}
	\end{frame}

	\begin{frame}
		\frametitle{Federated Learning - Issues}
		\begin{itemize}
			\item Vulnerable to central server going down
			\item Requires that every node has direct access to the server
			\item Few slow nodes slow the whole process down
		\end{itemize}
	\end{frame}


	\begin{frame}
		\frametitle{Swarm Learning}
		\begin{itemize}
			\item Each node has a distinct model, called the \emph{local model}
			\begin{itemize}
				\item Every model approximates the \emph{global model}
			\end{itemize}
			\item Each node has its own dataset
			\begin{itemize}
				\item This dataset cannot be shared with any other nodes
			\end{itemize}
			\item The goal is to train the \emph{global model} using all available data
			\item No central server or node acting as a central server 
		\end{itemize}
	\end{frame}

	\begin{frame}
		\frametitle{Swarm Learning - How Does It Work?}
		\begin{itemize}
			\item Repeated Training Steps. Each node each step:
			\begin{enumerate}
				\item Perform training on the local model
				\item Send trained model to all neighbours
				\begin{itemize}
					\item This will get cached on the neighbour
				\end{itemize}
				\item New local model is the combination of all neighbours most recent local models
			\end{enumerate}
		\item During step 3, the cached models are used to prevent the node having to wait for responses
		\end{itemize}
	\end{frame}

	\begin{frame}
		\frametitle{Swarm Learning - Specifics}
		\begin{itemize}
			\item Different combination methods
			\begin{itemize}
				\item Combine by average
				\item Combine with learning rate
			\end{itemize}
			\item Only combine neighbours who have done more training than this node
			\item Wait for certain number of neighbours to catch up
		\end{itemize}
	\end{frame}

	\begin{frame}
		\frametitle{Swarm Learning vs Issues of Federated Learning}
		\begin{itemize}
			\item \sout{Vulnerable to central server going down}
			\begin{itemize}
				\item No central server - to stop training you would have to take out every node
			\end{itemize}
			\item \sout{Requires that every node has direct access to the server}
			\begin{itemize}
				\item Swarm learning can function on sparse networks of nodes
			\end{itemize}
			\item \sout{Few slow nodes slow the whole process down}
			\begin{itemize}
				\item You never have to wait for a node due to caching
			\end{itemize}
		\end{itemize}
	\end{frame}

	\begin{frame}
		\frametitle{Swarm Learning - Performance}
		\begin{itemize}
			\item Many different configurations of the algorithm, can drastically affect performance
			\begin{itemize}
				\item In following plots only top 3 in a category have been shown
				\item For example may find top 3 configurations by 'area under graph'
			\end{itemize}
			\item Following plots are \emph{accuracy} of classifying MNIST Fashion, and x axis is number of epochs trained
			\begin{itemize}
				\item To make the problem a little harder each node only has 10 percent of the dataset
			\end{itemize}
			\item Federated Averaging is also shown
		\end{itemize}
	\end{frame}

	\begin{frame}
		\includegraphics[width = \textwidth]{fill}
	\end{frame}

	\begin{frame}
		\includegraphics[width = \textwidth]{fill_zoom}
	\end{frame}

	\begin{frame}
		\includegraphics[width = \textwidth]{stability}
	\end{frame}
	
	\begin{frame}
		\includegraphics[width = \textwidth]{stability_zoom}
	\end{frame}

	\begin{frame}
		\frametitle{Swarm Learning - Performance}
		\begin{itemize}
			\item Those plots were in a densely connected situation
			\begin{itemize}
				\item I have not got round to testing sparse networks yet
			\end{itemize}
		\item It would also be ideal to test more datasets but I don't think I will be able to due to time constraints
		\end{itemize}
	\end{frame}

	\begin{frame}
		\frametitle{Conclusion}
		\begin{itemize}
			\item Swarm Learning is a promising machine learning algorithm for training a model on data distributed on private data islands
			\item It addresses some of the issues with Federated Averaging, one of the current techniques
			\item It does not perform quite as well as Federated Averaging in a densely connected network
		\end{itemize}
	\end{frame}

	\begin{frame}
		\frametitle{Josh Pattman}
		Thanks for listening!
		\begin{itemize}
			\item LinkedIn: \textbf{linkedin.com/in/josh-pattman}
			\item GitHub: \textbf{github.com/JoshPattman}
		\end{itemize}
	\end{frame}
	
\end{document}